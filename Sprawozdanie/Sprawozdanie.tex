\documentclass[12pt, a4paper]{article}

\usepackage[T1]{fontenc}
\usepackage[utf8]{inputenc}
\usepackage[english, polish]{babel}
\usepackage{polski}
\usepackage{graphicx}
\usepackage[export]{adjustbox}
\usepackage{amsmath}
\usepackage{pdfpages}
\usepackage{enumerate}
\usepackage{siunitx}


\begin{titlepage}
\author{
	\\ Krystian Chachuła
	\\ Marcin Dolicher
	\\
	\\ AIR Semestr V
}
\title{
    \quad \quad \quad \quad \quad SMS
	\newline
	Sprawozdanie z Projektu I		
}
\date{}
\end{titlepage}

\begin{document}
\maketitle
\newpage
\tableofcontents

\newpage
\section{Zagadnienia i założenia projektowe}
Postawione przed nami zadanie polegało na zaprojektowaniu regulatora PID i DMC, które sterują obiektem zrealizowanym na mikrokontrolerach z serii STM32. Powinniśmy tak manipulować sygnałem wejściowym procesu $u$, aby wartość sygnału wyjściowego procesu (regulowanego) $y$ była możliwie bliska wartośći zadanej $y^{zad}$. Wartość uchybu $e=y^{zad} - y$ powinna być jak najmniejsza. Wyniki uzyskane podczas eksperymentów zostaną porównane i poddane krytycznej weryfikacji. 
\section{Algorytm PID}

\subsection{Omówienie implementacji}

\section{Algorytm DMC}

\section{Porównanie najlepszych realizacji PID i DMC}

\subsection{f) }
\begin{figure}[h!]
%\includegraphics[scale=0.55, center]{sciezka do pliku}
\end{figure}

\section{Testowanie}

\section{Wnioski}





















































\end{document}

