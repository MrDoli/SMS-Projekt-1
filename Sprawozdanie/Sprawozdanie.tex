\documentclass[12pt, a4paper]{article}

\usepackage[T1]{fontenc}
\usepackage[utf8]{inputenc}
\usepackage[english, polish]{babel}
\usepackage{polski}
\usepackage{graphicx}
\usepackage[export]{adjustbox}
\usepackage{amsmath}
\usepackage{pdfpages}
\usepackage{enumerate}
\usepackage{siunitx}


\begin{titlepage}
\author{
	\\ Krystian Chachuła
	\\ Marcin Dolicher
	\\
	\\ AIR Semestr V
}
\title{
    \quad \quad \quad \quad \quad SMS
	\newline
	Sprawozdanie z Projektu I		
}
\date{}
\end{titlepage}

\begin{document}
\maketitle
\newpage
\tableofcontents

\newpage
\section{Zagadnienia i założenia projektowe}
Postawione przed nami zadanie polegało na zaprojektowaniu regulatora PID i DMC, które sterują obiektem zrealizowanym na mikrokontrolerach z serii STM32. Powinniśmy tak manipulować sygnałem wejściowym procesu $u$, aby wartość sygnału wyjściowego procesu (regulowanego) $y$ była możliwie bliska wartośći zadanej $y^{zad}$. Wartość uchybu $e=y^{zad} - y$ powinna być jak najmniejsza. Wyniki uzyskane podczas eksperymentów zostaną porównane i poddane krytycznej weryfikacji. 
\section{Algorytm PID}

\subsection{Omówienie implementacji}

Tradycyjnie regulację za pomocą algorytmu PID realizujemy za pomocą trzech członów proporcjonalnego, całkującego i różniczkującego. Człon proporcjonalny powoduje wzrost wartości sterowania wraz z wzrostem uchybu, całkujący zwiększa wartość sygnału sterującego wraz z akumulowanym uchybem, a dla różniczkującego wraz z wzrostem uchybu, wzrasta wartość sygnału sterującego. 
\\\\Implementacji algorytmu dokonaliśmy w plikach \verb|Pid.c| i \verb|Pid.h|. Parametry PID-a zostały w kodzie zaprezentowane jako struktura \textit{Pid}. W pliku \verb|main.c| zadajemy wartości odpowiednim parametrom z struktury PID. Wymagane obliczenia w algorytmie są realizowane za pomocą funkcji \verb|float pidCe(Pid *pid, float pv)|, której argumentami są struktura z wartościami naszego PID-a i zmienna \textit{pv - process value}, czyli naszą wartość zadaną, a funkcja zwraca nam sygnał sterujący. 

~\\ W każdym wywołoniu funkcji dokonujemy następujących obliczeń: 
\begin{enumerate}
\item Wyliczamy uchyb na podstawie wzoru: $e(k) = y^{zad}(k) - y(k)$
\item Wartość członu proporcjonalnego $u(P)=Ke(k)$
\item Wartość członu całkującego $u_{I}(k) = u_{I}(k-1) + \dfrac{K}{T_{I}}T\dfrac{e(k-1)+e(k)}{2}$
\item Wartość członu różniczkującego $u_{D}(k) = KT_{D}\dfrac{e(k)-e(k-1)}{T}$
\item Następuje zapisanie wartości z stanu k jako wartości dla stanu k-1 (w naszym kodzie zmienne z poprzedniego stanu wyrażone są za pomocą przedrostka prev)
\end{enumerate}

Oprócz tych kroków do naszego algorytmu zastosowaliśmy rozwiązanie anti-windup. Rozwiązania tego używamy w przypadku gdy zmienna sterowania osiąga wartość graniczną urządzenia wykonawczego. Wiemy, że nie ma sensu zadawać większej wartości sygnału sterowania niż element wykonawczy jest w stanie zrealizować. W takiej sytuacji przerywamy pętlę sprzężenia zwrotnego i system zaczyna pracę w pętli otwartej. Takie rozwiązanie zapobiega ,,nawijaniu'' członu całkującego, czyli osiąganiu nadzwyczaj dużych wartości członu całkującego co prowadzi do ogromnego spowolnienia działania regulatora, a w skrajnych przypadkach do jego rozregulowania. 

~\\

Odp skokowa !!!!!!!!!!!!!!!!??????????????????????
Dostrajanie regulatora odbywało się na zasadzie pozyskiwania odpowiedz skokowej. Obserwując w jaki sposób sygnał sterujący generowany przez regulator osiąga wartość zadaną podczas skoku dokonywaliśmy oceny regulacji. W ten sposoób wybieraliśmy najlepsze nastawy dla regulatora. 

\subsection{Wyznaczanie nastawów regulatora metodą Zieglera-Nicholsa}

Przbieg strojenia regulatora przy użyciu metody Zieglera-Nicholsa
\begin{enumerate}
\item Implementujemy regulator typu P.
\item Wartość wzmocnienia K dobieramy tak aby wyjście obiektu regulacji miało charakter oscylacyjny (nierosnący, niemalejący). Przyjmujemy wzmocnienie krytyczne $K_{u} = K$. Odczytujemy jeszcze okres oscylacji $T_{u}$.
\item Używając tabelki wyliczamy parametry $K, T_{I}, T_{D}$ w zależności od regulatora który chcemy stosować P, PI, PID. My oczywiście wybiermay wzory dla PID. 

\end{enumerate}

\section{Algorytm DMC}

\section{Porównanie najlepszych realizacji PID i DMC}

\subsection{f) }
\begin{figure}[h!]
%\includegraphics[scale=0.55, center]{sciezka do pliku}
\end{figure}

\section{Testowanie}

\section{Wnioski}





















































\end{document}

